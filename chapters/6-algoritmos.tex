\section{Operadores Relacionales}
Cada operador físico implementa un operador relacional lógico, y está orientado a desempeñar una tarea específica.

Existen múltiples operadores físicos, como la selección ($\sigma$), la proyección ($\pi$), y sorting ($\tau$).

\subsection{Sorting}
Se considera que un conjunto de tuplas $\{ t_1, \ldots, t_n \}$ está ordenado respecto a una secuencia $(k_1, \ldots, k_m)$ si
\[ t_i \leq_{k_1, \ldots, k_m} t_{i+1} \Rightarrow t_i(k_1) < t_{i+1}(k_1) \vee (t_i(k_1) = t_{i+1}(k_1) \wedge t_i \leq_{k_2, \ldots, k_m} t_{i+1}) \]

Existen varios algoritmos para ordenar. Nos vamos a enfocar en MergeSort. MergeSort se basa en ordenar un par de listas que ya están ordenadas, para mezclarlas y dejarlas como una sola lista ordenada.

\begin{paracol}{2}
  \begin{algorithm}[h]
    \caption{merge($L_1$, $L_2$)}
    \KwIn{Dos listas ordenadas $L_1$ y $L_2$}
    % \KwOut{Una lista ordenada $L$ con los elementos de $L_1$ y $L_2$}
    \KwOut{Una lista ordenada $L$}
    $L$ = lista vacía\;
    \While{$L_1$ y $L_2$ son ambas no vacias}{
      Quitar el menor elemento $m$ entre $L_1$ y $L_2$\;
      Agregar $m$ al final de $L$
    }
    \uIf{$L_1$ está vacía}{
      Remover elementos de $L_2$ y agregar al final de $L$.\;
    }
    \ElseIf{$L_2$ está vacía}{
      Remover elementos de $L_1$ y agregar al final de $L$.\;
    }
    
    \Return{$L$}
  \end{algorithm}
  
  \switchcolumn
  
  \begin{algorithm}[h]
    \caption{mergesort($L$)}
    \KwIn{Una lista de numeros $L = a_1, \ldots, a_n$.}
    \KwOut{$L$ en orden creciente.}
    
    \eIf{$n = 1$}{
      \Return{$L$}
    }{
      $m = \lfloor n/2 \rfloor$\;
      $L_1 = a_1, a_2, \ldots, a_m$\;
      $L_2 = a_{m+1}, a_{m+2}, \ldots, a_n$\;
      \Return{merge(mergesort($L_1$), mergesort($L_2$))}
    }
  \end{algorithm}
\end{paracol}
\pagebreak

Existe una alternativa llamada External Merge Sort, la cual hace una tarea similar a Merge Sort, pero guardando los elementos del disco en un buffer de memoria, para en ese momento fusionar ambas listas y guardarlas al disco.

\begin{algorithm}[h]
  \caption{externalMergeSort($P$)}
  \KwIn{Páginas $P = \{ p_1, \ldots, p_n \}$ con tuplas.}
  \KwOut{Lista de páginas $P'$ con las tuplas ordenadas.}
  \ForEach{$p \in P$}{
    Leemos $p$ a memoria.\;
    Ordenamos $p$ en $r$.\;
    Escribimos $r$ en disco.
  }
  $R = \{ r_1, \ldots, r_n \}$
  \While{$|R| > 1$}{
    $R' = \emptyset$\;
    \ForEach{$r_1, r_2 \in R$}{
      Leemos $r_1$ y $r_2$ a memoria. ($R = R - \{r_1, r_2\}$)\;
      Mergeamos $r_1$ y $r_2$ ($r = \text{merge}(r_1, r_2)$)\;
      Escribimos $r$ en disco ($R' = R' \cup \{ r \}$)
    }
    $R = R'$
  }
  \Return{Único run en $R$}
\end{algorithm}

El costo I/O de este algoritmo, considerando $N$ el número de páginas del archivo, es de $2 \cdot N \cdot (1 + \lceil \log_2(N))$. Esto significa que un archivo de 8GB con 1.048.576 páginas tomará 1.2 horas en hacer sort. Esto es mucho.

El algoritmo puede ser optimizado si es que se aumenta el número de runs en el merge, se reduce el número de runs iniciales y se evita escribir el último resultado. Con estas cosas el algoritmo queda así
\pagebreak

\begin{algorithm}[h]
  \caption{externalMergeSort($P$) optimizado}
  \KwIn{Páginas $P = \{ p_1, \ldots, p_n \}$ con tuplas y $B+1$ páginas en buffer.}
  \KwOut{Lista de páginas $P'$ con las tuplas ordenadas.}
  \While{$P \not= \emptyset$}{
    Leemos $\{p_1, \ldots, p_{B+1}\}$ a memoria ($P = P - \{ p_1, \ldots, p_{B+1} \}$)\;
    Ordenamos $\{p_1, \ldots, p_{B+1}\}$ con quicksort y creamos un run $r$\;
    Escribimos $r$ en disco
  }
  $R = \{ r_1, \ldots, r_{\frac{n}{B+1}} \}$\;
  \ForEach{$|R| > 1$}{
    $R' = \emptyset$\;
    \While{$R \not= \emptyset$}{
      Leemos $r_1, \ldots, r_B$ a memoria ($R = R - \{ r_1, \ldots, r_B \}$)\;
      Hacemos merge de $r_1, \ldots, r_B$ ($r = merge(r_1, \ldots, r_B)$)\;
      Escribimos $r$ en disco ($R' = R' \cup \{ r \}$)
    }
    $R = R'$
  }
  \Return{Único run en $R$}
\end{algorithm}

Ahora su costo total en I/O es de $2 \cdot N \cdot (1 + \lceil \log_B(\lceil \frac{N}{B+1} \rceil) \rceil)$. Con eso, el mismo ejemplo de antes, y considerando un buffer de 2GB de RAM, es posible ordenar en tan solo 6.7 minutos.

Como nota, los parámetros de costo de sorting son los siguientes
\begin{itemize}
  \item $\text{cost}(\tau(R)) = \text{cost}(R) + 2 \cdot \text{pages}(R)$
  \item $\text{pages}(\tau(R)) = \text{pages}(R)$
  \item $|\tau(R)| = |R|$
  \item $\text{rsize}(\tau(R)) = \text{rsize}(R)$
\end{itemize}
\pagebreak

\subsection{Selección}
La selección es el proceso de elegir ciertos datos de una relación que calcen con un cierto criterio.

La selección se puede dar en varios casos: sin índices, con índice primario, con índice secundario y con distintos filtros. Cada uno de estos casos genera pequeños cambios en el algoritmo, tal como se mostrará ahora.

Para el caso sin índices, la única forma de filtrar es leer cada una de las tuplas, y retornarla si satisface el predicado.
\begin{paracol}{2}
  \begin{algorithm}
    \caption{select($p$, $R$) no index}
    \KwIn{Predicado $p$ y relación $R$}
    \DontPrintSemicolon
    \SetKwFunction{FOpen}{open}
    \SetKwFunction{FClose}{close}
    \SetKwFunction{FNext}{next}
    \SetKwProg{Fn}{Function}{:}{}
  
    \Fn{\FOpen{}}{
      $R$.open()
    }
    \Fn{\FClose{}}{
      $R$.close()
    }
  
    \Fn{\FNext{}}{
      $t$ = $R$.next()\;
      \While{$t \not= \texttt{NULL}$}{
        \If{$p(t) = \texttt{true}$}{
          \Return{$t$}
        }
        $t$ = $R$.next()
      }
      \Return{\texttt{NULL}}    
    }
  \end{algorithm}

  \switchcolumn

  Costo y parámetros de $\sigma_p(R)$
  \begin{itemize}
    \item $\text{cost}(\sigma_p(R)) = \text{cost}(R)$
    \item $\text{pages}(\sigma_p(R)) = \text{sel}_p(R) \cdot \text{pages}(R)$
    \item $|\sigma_p(R)| = \text{sel}_p(R) \cdot |R|$
  \end{itemize}
\end{paracol}


Para el caso de índice primario, se busca la primera tupla que satisface $A = v$. Luego se retornan las siguientes tuplas haciendo next en el índice.
\begin{paracol}{2}
  \begin{algorithm}
    \caption{select($p$, $R$) primary index}
    \KwIn{Predicado $p := A=v$, relación $R$ e índice $I$}
    \DontPrintSemicolon
    \SetKwFunction{FOpen}{open}
    \SetKwFunction{FClose}{close}
    \SetKwFunction{FNext}{next}
    \SetKwProg{Fn}{Function}{:}{}
  
    \Fn{\FOpen{}}{
      $I$.open()\;
      $I$.search($A=v$)
    }
    \Fn{\FClose{}}{
      $I$.close()
    }
  
    \Fn{\FNext{}}{
      $t$ = $I$.next()\;
      \If{$t \not= \texttt{NULL}$}{
        \Return{$t$}
      }
      \Return{\texttt{NULL}}    
    }
  \end{algorithm}

  \switchcolumn

  Costo y parámetros de $\sigma_p(R)$
  \begin{itemize}
    \item $\text{cost}(\sigma_p(R)) = \text{cost}(I) + \text{sel}_p(R) \cdot \text{pages}(R)$
    \item $\text{pages}(\sigma_p(R)) = \text{sel}_p(R) \cdot \text{pages}(R)$
    \item $|\sigma_p(R)| = \text{sel}_p(R) \cdot |R|$
  \end{itemize}
\end{paracol}
\pagebreak

Para el caso de índice secundario, se busca la primera tupla que satisface $A = v$. Luego, por cada data entry $k*$ que satisface $A = v$, se busca la pagina en $k*.\text{RID}$. Finalmente, se retorna la tupla con RID igual a $k*.\text{RID}$.
\begin{paracol}{2}
  \begin{algorithm}
    \caption{select($p$, $R$) secondary index}
    \KwIn{Predicado $p := A=v$, relación $R$ e índice $I$}
    \DontPrintSemicolon
    \SetKwFunction{FOpen}{open}
    \SetKwFunction{FClose}{close}
    \SetKwFunction{FNext}{next}
    \SetKwProg{Fn}{Function}{:}{}
  
    \Fn{\FOpen{}}{
      $I$.open()\;
      $I$.search($A=v$)
    }
    \Fn{\FClose{}}{
      $I$.close()
    }
  
    \Fn{\FNext{}}{
      $k*$ = $I$.next()\;
      \If{$k* \not= \texttt{NULL}$}{
        \Return{$R.\text{get}(k*.\text{RID})$}
      }
      \Return{\texttt{NULL}}    
    }
  \end{algorithm}

  \switchcolumn

  Costo y parámetros de $\sigma_p(R)$
  \begin{itemize}
    \item $\text{cost}(\sigma_p(R)) = \text{cost}(I) + \text{sel}_p(R) \cdot |R|$
    \item $\text{pages}(\sigma_p(R)) = \text{sel}_p(R) \cdot \text{pages}(R)$
    \item $|\sigma_p(R)| = \text{sel}_p(R) \cdot |R|$
  \end{itemize}
\end{paracol}


\subsection{Proyección}
La proyección tiene como fin mostrar cierta columna $L$ de una relación $R$.
\begin{paracol}{2}
  \begin{algorithm}
    \caption{project($p$, $R$)}
    \KwIn{Predicado $p := A=v$, relación $R$ e índice $I$}
    \DontPrintSemicolon
    \SetKwFunction{FOpen}{open}
    \SetKwFunction{FClose}{close}
    \SetKwFunction{FNext}{next}
    \SetKwProg{Fn}{Function}{:}{}
  
    \Fn{\FOpen{}}{
      $I$.open()\;
      $I$.search($A=v$)
    }
    \Fn{\FClose{}}{
      $I$.close()
    }
  
    \Fn{\FNext{}}{
      $t$ = $R$.next()\;
      \If{$t \not= \texttt{NULL}$}{
        \Return{$t$.project($L$)}
      }
      \Return{\texttt{NULL}}    
    }
  \end{algorithm}

  \switchcolumn

  Costo y parámetros de $\pi_L(R)$
  \begin{itemize}
    \item $\text{cost}(\pi_L(R)) = \text{cost}(R)$
    \item $\text{pages}(\pi_L(R)) = \frac{\text{rsize}(\pi_L(R))}{\text{rsize}(R)} \cdot \text{pages}(R)$
    \item $|\pi_L(R)| = |R|$
    \item $\text{rsize}(\pi_L(R)) = \sum_{\text{att} \in L} \mathbb{E}(|\pi_{\text{att}}(R)|)$
  \end{itemize}
\end{paracol}

\pagebreak

\subsection{Join}
Nested loops join
\begin{paracol}{2}
  \begin{algorithm}
    \caption{join($R$, $S$)}
    \KwIn{Predicado $p$, operadores $R$ y $S$}
    \DontPrintSemicolon
    \SetKwFunction{FOpen}{open}
    \SetKwFunction{FClose}{close}
    \SetKwFunction{FNext}{next}
    \SetKwProg{Fn}{Function}{:}{}
  
    \Fn{\FOpen{}}{
      $R$.open()\;
      $S$.open()\;
      $r$ = $R$.next()
    }
    \Fn{\FClose{}}{
      $R$.close()\;
      $S$.close()
    }
  
    \Fn{\FNext{}}{
      \While{$r \not= \texttt{NULL}$}{
        $s$ = $S$.next()\;
        \If{$s = \texttt{NULL}$}{
          $S$.close()\;
          $r$ = $R$.next()\;
          $S$.open()
        }
        \ElseIf{$(r,s)$ satisfacen $p$}{
          \Return{$(r,s)$}
        }
      }
      \Return{\texttt{NULL}}   
    }
  \end{algorithm}

  \switchcolumn

  Costo y parámetros de $R \bowtie_p S$
  \begin{itemize}
    \item $\text{cost}(R \bowtie_p S) = \text{cost}(R) + |R| \cdot \text{cost}(S)$
    \item $\text{pages}(R \bowtie_p S) = \text{sel}_p(R \times S) \cdot \text{pages}(R) \cdot \text{pages}(S)$
    \item $|R \bowtie_p S| = \text{sel}_p(R \times S) \cdot |R| \cdot |S|$
    \item $\text{rsize}(R \bowtie_p S) = \text{rsize}(R) + \text{rsize}(S)$
  \end{itemize}
\end{paracol}

\section{Estimación de Costos}
Para estimar que tan computacionalmente costosa puede llegar a ser una operación, se tienden a usar los siguientes parámetros
\begin{itemize}
  \item cost($R$): Costo en I/O para computar $R$\footnote{$R$: Relación que está participando en la operación.}
  \item pages($R$): Cantidad de páginas necesarias para almacenar $R$.
  \item $|R|$: Cantidad de tuplas en $R$.
  \item rsize($R$): Tamaño de una tupla en promedio en $R$.
  \item distinct($R$): Cantidad de elementos distintos en $R$.
  \item $\text{distinct}_a(R)$: Cantidad de elementos distintos en $R$, según la columna $a$.
  \item $|\text{page}|$: Tamaño de una página.
  \item $\text{sel}_p(R)$: Fracción de tuplas en $R$ que satisfacen $p$.\footnote{Se habla de alta selectividad cuando pocos elementos de $R$ satisfacen a $p$, osea, cuando $\text{sel}_p(R) \sim 0$}
\end{itemize}
